\documentclass[]{article}
\usepackage{amsfonts, amssymb}
\usepackage{amsmath}
\usepackage{mathtools}
\usepackage{algorithmic}
\usepackage{float}
\usepackage{graphicx}
\usepackage{enumitem}

\title{Question 39.2023}
\author{Anupama Kulshreshtha \\ EE22BTECH11009}
\date{}
\begin{document}
\maketitle
\providecommand{\pr}[1]{\ensuremath{\Pr\left(#1\right)}}
\providecommand{\prt}[2]{\ensuremath{p_{#1}^{\left(#2\right)} }}        % own macro for this question
\providecommand{\qfunc}[1]{\ensuremath{Q\left(#1\right)}}
\providecommand{\sbrak}[1]{\ensuremath{{}\left[#1\right]}}
\providecommand{\lsbrak}[1]{\ensuremath{{}\left[#1\right.}}
\providecommand{\rsbrak}[1]{\ensuremath{{}\left.#1\right]}}
\providecommand{\brak}[1]{\ensuremath{\left(#1\right)}}
\providecommand{\lbrak}[1]{\ensuremath{\left(#1\right.}}
\providecommand{\rbrak}[1]{\ensuremath{\left.#1\right)}}
\providecommand{\cbrak}[1]{\ensuremath{\left\{#1\right\}}}
\providecommand{\lcbrak}[1]{\ensuremath{\left\{#1\right.}}
\providecommand{\rcbrak}[1]{\ensuremath{\left.#1\right\}}}
\newcommand{\sgn}{\mathop{\mathrm{sgn}}}
\providecommand{\abs}[1]{\left\vert#1\right\vert}
\providecommand{\res}[1]{\Res\displaylimits_{#1}} 
\providecommand{\norm}[1]{\left\lVert#1\right\rVert}
%\providecommand{\norm}[1]{\lVert#1\rVert}
\providecommand{\mtx}[1]{\mathbf{#1}}
\providecommand{\mean}[1]{E\left[ #1 \right]}
\providecommand{\cond}[2]{#1\middle|#2}
\providecommand{\fourier}{\overset{\mathcal{F}}{ \rightleftharpoons}}
\newenvironment{amatrix}[1]{%
  \left(\begin{array}{@{}*{#1}{c}|c@{}}
}{%
  \end{array}\right)
}
%\providecommand{\hilbert}{\overset{\mathcal{H}}{ \rightleftharpoons}}
%\providecommand{\system}{\overset{\mathcal{H}}{ \longleftrightarrow}}
	%\newcommand{\solution}[2]{\textbf{Solution:}{#1}}
\newcommand{\solution}{\noindent \textbf{Solution: }}
\newcommand{\cosec}{\,\text{cosec}\,}
\providecommand{\dec}[2]{\ensuremath{\overset{#1}{\underset{#2}{\gtrless}}}}
\newcommand{\myvec}[1]{\ensuremath{\begin{pmatrix}#1\end{pmatrix}}}
\newcommand{\mydet}[1]{\ensuremath{\begin{vmatrix}#1\end{vmatrix}}}
\newcommand{\myaugvec}[2]{\ensuremath{\begin{amatrix}{#1}#2\end{amatrix}}}
\providecommand{\rank}{\text{rank}}
\providecommand{\pr}[1]{\ensuremath{\Pr\left(#1\right)}}
\providecommand{\qfunc}[1]{\ensuremath{Q\left(#1\right)}}
	\newcommand*{\permcomb}[4][0mu]{{{}^{#3}\mkern#1#2_{#4}}}
\newcommand*{\perm}[1][-3mu]{\permcomb[#1]{P}}
\newcommand*{\comb}[1][-1mu]{\permcomb[#1]{C}}
\providecommand{\qfunc}[1]{\ensuremath{Q\left(#1\right)}}
\providecommand{\gauss}[2]{\mathcal{N}\ensuremath{\left(#1,#2\right)}}
\providecommand{\diff}[2]{\ensuremath{\frac{d{#1}}{d{#2}}}}
\providecommand{\myceil}[1]{\left \lceil #1 \right \rceil }
\newcommand\figref{Fig.~\ref}
\newcommand\tabref{Table~\ref}
\newcommand{\sinc}{\,\text{sinc}\,}
\newcommand{\rect}{\,\text{rect}\,}
%%
%	%\newcommand{\solution}[2]{\textbf{Solution:}{#1}}
%\newcommand{\solution}{\noindent \textbf{Solution: }}
%\newcommand{\cosec}{\,\text{cosec}\,}
%\numberwithin{equation}{section}
%\numberwithin{equation}{subsection}
%\numberwithin{problem}{section}
%\numberwithin{definition}{section}
%\makeatletter
%\@addtoreset{figure}{problem}
%\makeatother

%\let\StandardTheFigure\thefigure
\let\vec\mathbf

Let $X_1$, $X_2$, ... , $X_n$ be a random sample of size $n$ from a population having probability density function
\begin{align}
f(x; \mu) =
\begin{cases}
e^{-(x-\mu)}, & \text{if } \mu \leq x < \infty \\
0, & \text{otherwise,} 
\end{cases}
\end{align}
where $\mu \in \mathbb{R}$ is an unknown parameter. If $\hat{M}$ is the maximum likelihood estimator of the median of $X_1$, then which one of the following statements is true?
\begin{enumerate}[label=\Alph*)]
  \item P($\hat{M} \leq 2$) = $1 - e^{-n(1-\log_e 2)}$ if $\mu = 1$
  \item P($\hat{M} \leq 1$) = $1 - e^{-n \log_e 2}$ if $\mu = 1$
  \item P($\hat{M} \leq 3$) = $1 - e^{-n(1-\log_e 2)}$ if $\mu = 1$
  \item P($\hat{M} \leq 4$) = $1 - e^{-n(2\log_e 2-1)}$ if $\mu = 1$
\end{enumerate}
\solution
For continuous random variable X, median M is such that,
\begin{align}
P(X \leq M) &= 0.5
\end{align}
The pdf of X is given by,
\begin{align}
f(x; \mu) =
\begin{cases}
e^{-(x-\mu)}, & \text{if } \mu \leq x < \infty \\
0, & \text{otherwise,} 
\end{cases}
\end{align}
Hence, cdf is given by
\begin{align}
F(x; \mu) &= \int_{\mu}^{x} e^{-(t-\mu)}dt\\ 
&= e^{\mu}[-e^{-x} + e^{-\mu}]\\
&= 1 - e^{-(x-\mu)}
\end{align}
Now,
\begin{align}
F(x; \mu) &= 0.5\\
\implies 1 - e^{-(M-\mu)} &= 0.5\\
\implies M &= \mu + \ln(2)
\end{align}
The MLE $\hat{M}$ of the median in a random sample is the middle value of the ordered sample. Since $M = \mu + \ln(2)$, the MLE of the median is $\hat{M} = \hat{\mu} + \ln(2)$, where $\hat{\mu}$ is the MLE of $\mu$.
Now, to calculate $P(\hat{M} \leq y)$,
\begin{align}
P(\hat{\mu} + \ln(2) \leq y) &= P(\hat{\mu} \leq y - \ln(2))
\end{align}
CDF of $\hat{\mu}$ is same as CDF of $\mu$, as MLE is biased estimator, so,
\begin{align}
F(\hat{\mu}) &= F(\mu)\\
\implies P(\hat{M} \leq y) &= 1 - e^{-n(y-M)}
\end{align}
When $y = 2$ and $\mu = 1$,
\begin{align}
P(\hat{M} \leq 2) &= 1 - e^{-n(1 - \log_e 2)}
\end{align}
Therefore, option (A) is correct.
\end{document}
